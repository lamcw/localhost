\section{Implementation}

\subsection{Dependencies}
This project uses the following dependencies.

\subsubsection{Celery}
When an auction session comes to an end, the system needs to check if there are
any bids under property item. If there are, the system should clean up all the
bids associated with the property item and adds a booking to the winner's
account. \emph{localhost} solves this by using a plug-in called Celery. It
provides a distributed task queue which delays the execution of certain jobs to
a particular time~\cite{celery-doc}. In our use case, we set up periodic tasks for each property
item so that it execute certain tasks when auction session is closed.

\subsubsection{channels-redis}
channels-redis is a Django Channels channel layer that uses Redis as its backing
store~\cite{channels-redis-doc}. This allows Django Channels to ``communicate''
across different ASGI instances, which is useful if we were to scale the
website to run on a distributed system.

\subsubsection{daphne}
Daphne is a HTTP, HTTP2 and WebSocket protocol server for ASGI and ASGI-HTTP
that was designed for Django Channels~\cite{daphne}. It mainly handles WebSocket requests in
our website, for example, bid requests, messaging and push notifications.

\subsubsection{Django}
The main framework this project runs on is Django. It is a high-level Python Web
framework that allows the team to quickly prototype a working product with clean
and pragmatic design. It is extremely scalable since it supports adding ``apps''
to a working site to extend its functionality~\cite{django}. % TODO

\subsubsection{Django Channels}
Django Channels extends Django's ability beyond HTTP to handle WebSocket
requests, chat protocols, etc. It allows communications between client and
server using a long-running connection, while at the same time preserve Django's
synchronous and easy-to-use nature~\cite{django-channels-doc}. All of the
real-time components of the project are implemented within the Django Channels
framework, including auctions, messaging and push notifications.

\subsubsection{django-celery-beat}
This extension allows us to store Celery \texttt{PeriodicTasks} in the database,
which could then be modified during
runtime~\cite{django-celery-beat-doc}. This is useful since we
might have to change tasks execution interval if auction sessions are modified
for a property item during run-time.

\subsubsection{django-widget-tweaks}
Widget-tweaks allows us to inject CSS attributes into form fields in
templates~\cite{widget-tweaks}. This means we can separate the business logic
in ``views'' \footnote{Django view functions or Django class-based views} and
``templates''. For example, if we want to add \texttt{class="form-control"}
into a form input without widget-tweaks, we would need to append the class names
in \texttt{\_\_init\_\_()} in the form class. With widget-tweaks, we can change the
field representation in templates with Django template syntaxes (liquid tags).

\subsubsection{Factory Boy}
To generate our test data for both unit-tests and performance benchmark, we use
Factory Boy to create random but reasonable test data. We uses an extension
instead of creating fixtures on our own so that it is easier to maintain and the
data is reproducible and consistent~\cite{factory-boy-doc}.

\subsubsection{googlemaps}
This is a Python Client for Google Maps Services which creates an abstraction
between the Google Maps raw JSON API and our Django
application~\cite{python-google-maps}. We uses this library to access the Maps
API in backend.

\subsubsection{gunicorn}
Gunicorn is a Python WSGI HTTP Server for UNIX~\cite{gunicorn}.

\subsubsection{psycopg2-binary}
psycopg2 is a binder/adapter between Python and PostgreSQL\@. It allows any Python
application to connect to a PostgreSQL database instance~\cite{psycopg2}.

\subsubsection{redis (Python)}
redis is a Python client for Redis~\cite{python-redis}.

\subsection{Licensing}
Below is a list of license used by all the dependencies.
\begin{longtable}{| p{.40\textwidth} | p{.20\textwidth} | p{.20\textwidth} |}
\hline
\textbf{Dependency} & \textbf{Version} & \textbf{License} \\ \hline
\textbf{Automat} & 0.7.0 & MIT \\
\textbf{Django} & 2.1.2 & BSD \\
\textbf{Faker} & 0.9.1 & MIT \\
\textbf{Pillow} & 5.3.0 & PIL \\
\textbf{PyHamcrest} & 1.9.0 & BSD \\
\textbf{Twisted} & 18.7.0 & MIT \\
\textbf{aioredis} & 1.1.0 & MIT \\
\textbf{amqp} & 2.3.2 & BSD \\
\textbf{asgiref} & 2.3.2 & BSD \\
\textbf{async-timeout} & 3.0.0 & Apache \\
\textbf{attrs} & 18.2.0 & MIT \\
\textbf{autobahn} & 18.9.2 & MIT \\
\textbf{billiard} & 3.5.0.4 & BSD \\
\textbf{celery} & 4.1.1 & BSD \\
\textbf{certifi} & 2018.8.24 & MPL-2.0 \\
\textbf{channels} & 2.1.3 & BSD \\
\textbf{channels-redis} & 2.3.0 & BSD \\
\textbf{chardet} & 3.0.4 & LGPL \\
\textbf{constantly} & 15.1.0 & MIT \\
\textbf{daphne} & 2.2.2 & BSD \\
\textbf{django-celery-beat} & 1.1.0 & BSD \\
\textbf{django-widget-tweaks} & 1.4.3 & MIT \\
\textbf{ephem} & 3.7.6.0 & LGPL \\
\textbf{factory-boy} & 2.11.1 & MIT \\
\textbf{googlemaps} & 3.0.2 & Apache \\
\textbf{gunicorn} & 19.9.0 & MIT \\
\textbf{hiredis} & 0.2.0 & BSD \\
\textbf{hyperlink} & 18.0.0 & MIT \\
\textbf{idna} & 2.7 & BSD-like \\
\textbf{incremental} & 17.5.0 & MIT \\
\textbf{kombu} & 4.2.1 & BSD \\
\textbf{msgpack} & 0.5.6 & Apache \\
\textbf{psycopg2-binary} & 2.7.5 & LGPL \\
\textbf{python-dateutil} & 2.7.3 & Dual \\
\textbf{pytz} & 2018.5 & MIT \\
\textbf{redis} & 2.10.6 & MIT \\
\textbf{requests} & 2.19.1 & Apache \\
\textbf{six} & 1.11.0 & MIT \\
\textbf{text-unidecode} & 1.2 & Artistic \\
\textbf{txaio} & 18.8.1 & MIT \\
\textbf{urllib3} & 1.23 & MIT \\
\textbf{vine} & 1.1.4 & BSD \\
\textbf{zope.interface} & 4.5.0 & ZPL \\
\hline
\end{longtable}

\subsection{Design}
\subsubsection{Architecture}
\subsubsection{User Interface}
\subsubsection{Database Design}
\subsubsection{Real Time Communication}
\subsubsection{Event Scheduling}
