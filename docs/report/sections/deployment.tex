\section{Deployment}

\emph{localhost} is deployed using Amazon Web Services on an Amazon Elastic
Compute Cloud instance with the following specifications.
\begin{lstlisting}
OS: Ubuntu 18.04.1 LTS x86_64
Host: HVM domU 4.2.amazon
Kernel: 4.15.0-1023-aws
CPU: Intel Xeon E5-2676 v3 (1) @ 2.4
GPU: Cirrus Logic GD 5446
Memory: 983MiB
\end{lstlisting}

\subsection{Amazon Web Services Configuration}

After creating an Amazon Elastic Compute Cloud instance through the Amazon Web
Services dashboard, the only port on the server accessible from the internet
is \lstinline{22 (SSH)}. For the server to accept web requests, ports
\lstinline{80 (HTTP)} and \lstinline{443 (SSL)} must be opened. This can be
done by navigating to the dashboard, selecting the running
\emph{Amazon Elastic Compute Cloud} instance, and selecting
\lstinline{launch-wizard}. A tab will then appear at the bottom of the page
and will contain a form that can open ports.

\subsection{Software}

While a number of scripts are provided to automate the installation and
configuration of \emph{localhost}, some manual configuration is required.
The configuration required varies between different Linux distribution so the
following steps are only supported on Ubuntu 18.04.1.

The default user account provided on an Amazon Elastic Compute Cloud instance
running Ubuntu 18.04.1 is \lstinline{ubuntu} and the following commands rely on
this assumption. Commands requiring sudo privileges are prepended with
\lstinline{#} and commands executed under user permissions are prepended with
\lstinline{$}.

\subsubsection{Update}

The initial state of the instance is likely out of date and should be brought
up to date by running the following command.
\begin{lstlisting}
# apt update && apt upgrade && apt dist-upgrade && apt autoremove
\end{lstlisting}

\subsubsection{Repository}

To ease updating, an SSH key will be generated to authenticate the server when
accessing the repository.
\begin{lstlisting}
$ ssh-keygen -t rsa -b 4096
Enter file in which to save the key (/home/ubuntu/.ssh/id_rsa):
Enter passphrase (empty for no passphrase):
Enter same passphrase again:
Your identification has been saved in id_rsa.
Your public key has been saved in id_rsa.pub.
\end{lstlisting}
After adding \lstinline{/home/ubuntu/.ssh/id_rsa.pub} to BitBucket the
repository can be cloned.
\begin{lstlisting}
$ git clone git@bitbucket.org:jtalowell/localhost.git /home/ubuntu/localhost
\end{lstlisting}

\subsubsection{Django \& Immediate Dependencies}

To install a virtual environment and manage dependencies the following
packages must be installed.
\begin{lstlisting}
# apt install python3-venv python3-pip
\end{lstlisting}
Django and its immediate dependencies can then be installed.
\begin{lstlisting}
$ cd /home/ubuntu/localhost
$ python3 -m venv venv
$ source venv/bin/activate
$ (venv) pip install -r requirements.txt
\end{lstlisting}

\subsubsection{PostgreSQL}

PostgreSQL must be installed and it's accompaning system service enabled.
\begin{lstlisting}
# apt install postgresql postgresql-contrib
# systemctl enable postgresql.service
# systemctl start postgresql.service
\end{lstlisting}
The database cluster can then be created.
\begin{lstlisting}
$ sudo -u postgres -i
[postgres]$ /usr/lib/postgresql/10/bin/initdb -D '/var/lib/postgresql/data'
\end{lstlisting}
A user to administer the database clusted must be created for access.
\begin{lstlisting}
[postgres]$ createuser --interactive
Enter name of role to add: ubuntu
Shall the new role be a superuser? (y/n) y
[postgres]$ exit
\end{lstlisting}
The database to use for the project can now be created.
\begin{lstlisting}
$ createdb localhost_db
$ sudo -u postgres -i
[postgres]$ psql
postgres=# \password ubuntu
Enter new password: [DB_PW]
Enter it again: [DB_PW]
postgres=# GRANT ALL PRIVILEGES ON DATABASE localhost_db TO ubuntu;
postgres=# ALTER ROLE ubuntu SET client_encoding TO ``utf8'';
postgres=# ALTER ROLE ubuntu SET default_transaction_isolation TO ``read committed'';
postgres=# \q
[postgres]$ exit
\end{lstlisting}

\subsubsection{Redis}

Ubuntu's package for Redis works without any extended configuration and its
services are automatically started and enabled.
\begin{lstlisting}
# apt install redis-server
\end{lstlisting}

\subsubsection{Daphne}

Daphne is installed as one of Django's immediate dependencies.
A service script is provided for installation.
\begin{lstlisting}
# ln -sf /home/ubuntu/localhost/deploy/systemd/daphne.service /etc/systemd/system/
# vim /etc/systemd/system/daphne.service
\end{lstlisting}
The environment variables for \lstinline{SECRET_KEY}, \lstinline{DB_USER}, and
\lstinline{DB_PW} must be set for Daphne to be able to operate. This
can be done by inserting the following lines in the service script.
\begin{lstlisting}
Environment="SECRET_KEY=YOUR KEY HERE"
Environment="DB_USER=ubuntu"
Environment="DB_PW=YOUR DB PASSWORD HERE"
\end{lstlisting}

\subsubsection{Gunicorn}

Gunicorn, like Daphne, is installed as one of Django's immediate dependencies.
A service script is provided for installation.
\begin{lstlisting}
# ln -sf /home/ubuntu/localhost/deploy/systemd/gunicorn.service /etc/systemd/system/
# vim /etc/systemd/system/gunicorn.service
\end{lstlisting}
The environment variables for \lstinline{SECRET_KEY}, \lstinline{DB_USER},
and \lstinline{DB_PW} must be set for Gunicorn to be able to operate.
This can be done by inserting the following lines in the service script.
\begin{lstlisting}
Environment="SECRET_KEY=YOUR KEY HERE"
Environment="DB_USER=ubuntu"
Environment="DB_PW=YOUR DB PASSWORD HERE"
\end{lstlisting}

\subsubsection{NGINX}

NGINX can be installed using the following commands.
\begin{lstlisting}
# apt install nginx
# systemd enable nginx
# systemd start nginx
\end{lstlisting}

The relevant configuration files can be installed by:
\begin{lstlisting}
# /home/ubuntu/localhost/deploy/init.sh [SECRET_KEY] [DB_USER] [DB_PW]
\end{lstlisting}

\emph{localhost} uses Let's Encrypt as the provider for it's SSL certificates.
The certificates can be installed using the Certbot tool which can be
installed and executed by the following commands. For this to work, the domain
name registered for \emph{localhost} should direct to the IP address of the
Amazon Elastic Compute Cloud instance.
\begin{lstlisting}
# apt install software-properties-common
# add-apt-repository ppa:certbot/certbot
# apt update
# apt install python-certbot-nginx
# certbot --nginx certonly
\end{lstlisting}

After installing the certificate the \lstinline{init.sh} script should be
executed again.
\begin{lstlisting}
# /home/ubuntu/localhost/deploy/init.sh [SECRET_KEY] [DB_USER] [DB_PW]
\end{lstlisting}
This script first ensures that the NGINX configuration script is installed
and restarts all the services that \emph{localhost} depends on.

\subsubsection{Maintenance}

There are two scripts that are provided to ease server upgrades and deployment
testing. The first script is \lstinline{rebuild.sh}. This script automates the
dropping, creation, migrations, and loading of test data for the database.
This is useful for quickly loading a usable set of test data without manually
creating listings and accounts. It's also extendable to restore backups.
\begin{lstlisting}
$ ./rebuild.sh [database name] -Mmsl [data name]
-M is short for --Makemigrations
-m is short for --migrate
-s is short for --server and should only be used on the server deployment
-l is short for --loaddata and loads both testdata and propertyimages
\end{lstlisting}

The second script has already been mentioned but remains relevent for future
updates. \lstinline{init.sh} should be ran every time any updates or upgrades
take place to restart the required services otherwise unexpected errors can
occur through caching.
\begin{lstlisting}
# ./init.sh [SECRET_KEY] [DB_USER] [DB_PW]
\end{lstlisting}
