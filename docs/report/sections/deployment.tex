\section{Deployment}

\emph{LocalHost} is deployed using \emph{Amazon Web Services} on an \emph{Amazon Elastic Compute Cloud} instance with the following specifications.
\begin{lstlisting}
OS: Ubuntu 18.04.1 LTS x86_64
Host: HVM domU 4.2.amazon
Kernel: 4.15.0-1023-aws
CPU: Intel Xeon E5-2676 v3 (1) @ 2.4
GPU: Cirrus Logic GD 5446
Memory: 983MiB
\end{lstlisting}

\subsection{Amazon Web Services Configuration}

After creating an instance, the only port on the server accessible from the internet is \lstinline{22 (SSH)}. For the server to accept web requests, ports \lstinline{80 (HTTP)} and \lstinline{443 (SSL)} must be opened. This can be done by navigating to the dashboard, selecting the running \emph{Amazon Elastic Compute Cloud} instance, and selecting \lstinline{launch-wizard}. A tab will then appear at the bottom of the page and will contain a form that can open ports.

\subsection{Software}

The default user account provided on an \emph{Amazon Elastic Compute Cloud} instance running \emph{Ubuntu} is \lstinline{ubuntu} and the following commands rely on this assumption. Commands requiring sudo privileges are prepended with \lstinline{#} and commmands executed under user permissions are prepended with \lstinline{$}.

\subsubsection{Repository}

To ease updating, an \lstinline{SSH} key will be generated to authenticate the server when accessing the repository.
\begin{lstlisting}
$ ssh-keygen -t rsa -b 4096
Enter file in which to save the key (/home/ubuntu/.ssh/id_rsa):
Enter passphrase (empty for no passphrase):
Enter same passphrase again:
Your identification has been saved in id_rsa.
Your public key has been saved in id_rsa.pub.
\end{lstlisting}
After adding \lstinline{/home/ubuntu/.ssh/id_rsa.pub} to \emph{BitBucket} the repository can then be cloned.
\begin{lstlisting}
$ git clone git@bitbucket.org:jtalowell/localhost.git /home/ubuntu/localhost
\end{lstlisting}

\subsubsection{Django \& Immediate Dependencies}

Django and its immediate dependencies must be installed.
\begin{lstlisting}
$ cd /home/ubuntu/localhost
$ python -m venv venv
$ source venv/bin/activate
$ (venv) pip install -r requirements.txt
\end{lstlisting}

\subsubsection{PostgreSQL}

\emph{PostgreSQL} must be installed and it's accompaning system service enabled.
\begin{lstlisting}
# apt install postgresql postgresql-contrib
# systemctl enable postgresql.service
# systemctl start postgresql.service
\end{lstlisting}
The database cluster must be created.
\begin{lstlisting}
$ sudo -u postgres -i
[postgres]$ /usr/lib/postgresql/10/bin/initdb -D '/var/lib/postgresql/data'
\end{lstlisting}
A user to administer the database clusted must be created.
\begin{lstlisting}
[postgres]$ createuser --interactive
Enter name of role to add: ubuntu
Shall the new role be a superuser? (y/n) y
[postgres]$ exit
\end{lstlisting}
The database to use for the project can now be created.
\begin{lstlisting}
$ createdb localhost_db
$ sudo -u postgres -i
[postgres]$ psql
postgres=# \password ubuntu
Enter new password: [DB_PW]
Enter it again: [DB_PW]
postgres=# GRANT ALL PRIVILEGES ON DATABASE localhost_db TO ubuntu;
postgres=# ALTER ROLE ubuntu SET client_encoding TO ``utf8'';
postgres=# ALTER ROLE ubuntu SET default_transaction_isolation TO ``read committed'';
postgres=# \q
[postgres]$ exit
\end{lstlisting}

\subsubsection{Redis}

\emph{Ubuntu}'s package for \emph{Redis} works without any extended configuration and its services are automatically started and enabled.
\begin{lstlisting}
# apt install redis-server
\end{lstlisting}

\subsubsection{Daphne}

\emph{Daphne} is installed as one of \emph{Django}'s immediate dependencies. The only thing left is to register a service script with \emph{Systemd}.
\begin{lstlisting}
# ln -sf /home/ubuntu/localhost/deploy/systemd/daphne.service /etc/systemd/system
# nano /etc/systemd/system/daphne.service
\end{lstlisting}
The environment variables for \lstinline{SECRET_KEY}, \lstinline{DB_USER}, and \lstinline{DB_PW} must be set prior to enabling. Below \lstinline{Environment="DJANGO_SETTINGS_MODULE=localhost.settings_production"} add:

\begin{lstlisting}
Environment="SECRET_KEY=YOUR KEY HERE"
Environment="DB_USER=ubuntu"
Environment="DB_PW=YOUR DB PASSWORD HERE"
\end{lstlisting}


\subsubsection{Gunicorn}
\subsubsection{NGINX}
\subsubsection{Maintenance}

